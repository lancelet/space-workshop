\documentclass[12pt,openany]{book}
\usepackage[a4paper,top=2.5cm,bottom=2.5cm,left=2cm,right=2cm]{geometry}

\usepackage{amsmath}

\usepackage[numbers]{natbib}

\usepackage[breaklinks,hidelinks]{hyperref}

\usepackage{booktabs}

\usepackage{tcolorbox}
\tcbuselibrary{skins}

\usepackage{xcolor}

\usepackage{fancyhdr}
\pagestyle{plain}

\renewcommand{\contentsname}{Table of Contents}

% Paragraph setting
\setlength{\parindent}{0em}
\setlength{\parskip}{1em}

% Set fonts
\usepackage{fontspec}
\setmainfont{Tex Gyre Pagella}
\setsansfont{Verdana}

% Chapter styling
\usepackage[grey]{quotchap}
\makeatletter
\renewcommand*{\chapnumfont}{%
  \usefont{T1}{\@defaultcnfont}{b}{n}\fontsize{80}{100}\selectfont% Default: 100/130
  \color{chaptergrey}%
}
\makeatother

% Custom mathematics
\renewcommand{\vec}[1]{\ensuremath{\mathbf{#1}}}
\newcommand{\vecdot}[1]{\ensuremath{\dot{\mathbf{#1}}}}

% Mathematical symbols environment
\newenvironment{symbols}
{%
  \begin{flushright}%
  \begin{tcolorbox}[enhanced,width=0.9\textwidth,colback=black!8!white,frame hidden]%
  \begin{tabular}{cp{0.8\textwidth}l}%
}
{%
  \end{tabular}%
  \end{tcolorbox}%
  \end{flushright}%
}
\newcommand{\unitsforce}{$\mathrm{N}$}
\newcommand{\unitsmass}{$\mathrm{kg}$}
\newcommand{\unitsacceleration}{$\mathrm{m}/\mathrm{s}^2$}
\newcommand{\unitstime}{$\mathrm{s}$}
\newcommand{\unitsmassflowrate}{$\mathrm{kg}/\mathrm{s}$}
\newcommand{\unitslength}{$\mathrm{m}$}
\newcommand{\unitsvelocity}{$\mathrm{m}/\mathrm{s}$}
\newcommand{\unitsradians}{$\mathrm{radians}$}

\title{
  The Haskell Space Program\\
  Planning an Orbital Launch
}
\author{
  Jonathan Merritt
  \and
  Luke Clifton
}
\date{Feb 2019}

\begin{document}
\maketitle
\tableofcontents

\chapter{Introduction}

\chapter{Rocket Equations of Motion}

The equations of motion of our rocket are a set of coupled, first-order, ordinary differential equations (ODEs). They describe the related rates of change of quantities that determine the rocket's motion. They provide a basis for all the simulation we will do later.

We will use a point mass model for the rocket. This means that we will model its linear motion, but not any angular motion. For a more complete consideration of these equations, consult \cite{gantmacher1950}.

\section{Governing Equations}

Newton's second law (Eq~\ref{eq:newton2}) is the main equation of motion for the rocket:
\begin{align}
  \vec{F} &= m \vecdot{v}  \label{eq:newton2}
\end{align}
\begin{symbols}
  \vec{F} & Force applied to the rocket. & \unitsforce \\
  $m$ & Instantaneous mass of the rocket, which varies over time. & \unitsmass \\
  \vecdot{v} & Acceleration of the rocket. & \unitsacceleration
\end{symbols}
The rocket burns fuel, causing its mass to decrease over time. The fuel creates an ejection of reaction mass from the rocket nozzle, which results in a thrust force. There is a relationship between mass flow rate, caused by burning the fuel, and the magnitude of thrust force exerted on the rocket. Modeling this relationship in full is a complicated hydrodynamics problem. However, we can approximate it as a linear relationship using a coefficient called specific impulse:
\begin{align}
  F_T = g_0\,I_{sp}\,\dot{m}
\end{align}
\begin{symbols}
  $F_T$ & Thrust force magnitude. & \unitsforce \\
  $g_0$ & Standard gravity (constant) 9.80665. & \unitsacceleration \\
  $I_{sp}$ & Specific impulse. & \unitstime \\
  $\dot{m}$ & Mass flow rate of the rocket. & \unitsmassflowrate
\end{symbols}

\section{Coupled ODEs}

We can write the first-order ODEs describing the rocket motion in the following form, based on the governing equations above:
\begin{align}
  \dot{m} &= c_{\dot{m}} \\
  \vecdot{x} &= \vec{v} \\
  \vecdot{v} &= \frac{\vec{F}}{m}
\end{align}
\begin{symbols}
  $m$ & Instantaneous mass of the rocket. & \unitsmass \\
  $c_{\dot{m}}$ & Control signal for the mass flow rate. & \unitsmassflowrate \\
  $\vec{x}$ & Position of the rocket. & \unitslength \\
  $\vec{v}$ & Velocity of the rocket. & \unitsvelocity \\ 
  $\vec{F}$ & Force applied to the rocket. & \unitsforce \\
\end{symbols}
The force is decomposed into terms describing thrust, gravity and drag:
\begin{align}
  \vec{F} &= \vec{F}_G(\vec{x})
  + \vec{F}_T(c_{\dot{m}}, c_{\theta})
  + \vec{F}_D(\ldots)
\end{align}
\begin{symbols}
  \vec{F} & Total force. & \unitsforce \\
  $\vec{F}_G$ & Force due to gravity. & \unitsforce \\
  $\vec{x}$ & Position of the rocket. & \unitslength \\
  $\vec{F}_T$ & Thrust force. & \unitsforce \\
  $c_{\dot{m}}$ & Control signal for the mass flow rate. & \unitsmassflowrate \\
  $c_{\theta}$ & Control signal for the thrust orientation. & \unitsradians \\
  $\vec{F}_D$ & Aerodynamic drag force. & \unitsforce \\
\end{symbols}
The remaining details will be described during the following chapters.

\chapter{Tsiolkovsky Rocket Equation and Delta-V}



\bibliographystyle{plainnat}
\bibliography{references.bib}

\end{document}
